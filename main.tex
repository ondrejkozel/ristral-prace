\documentclass[twoside,12pt]{article}%
% pro tisk po jedné straně papíru je potřebné odstranit volbu twoside
\usepackage{xdippo}
% \cestina % implicitní
% \slovencina
% \english
\pismo{LModern}
%\pismo{Constantia} % nic (=LModern), Academica, Baskerville, Bookman, Cambria, Comenia, Constantia, Palatino, Times
% \dvafonty % implicitně je \jedenfont
% \technika
% \beletrie % implicitní
\popiskyzkr
% \popisky % implicitní
\pagestyle{headings} % implicitní
\cislovat{2}
\brokenpenalty 10000
\usepackage{minted}
\usepackage[hidelinks]{hyperref}
\begin{document}

\titul{Tvorba jednoduchého řídicího informačního systému dopravní společnosti}{Ondřej Kozel}{Mgr.\,Naďa Horáková}{Brno 2021}

\prohlasenimuz{V~Brně dne \today}

\abstract{Production of a simple information system for transport company. Brno, 2022.}{There will be an abstract one day!}

\abstrakt{Tvorba jednoduchého řídicího informačního systému dopravní společnosti. Závěrečná maturitní práce. Brno, 2022.}
{Práce se zabývá procesem tvorby informačního systému pro dopravní společnost schopného spravovat a řídit vozidla, zastávky, linky, řidiče a další objekty veřejné dopravy.
Jejím hlavním cílem je vytvoření produktu, který by obstál při nasazení v menším dopravním podniku s nižšími nároky na celkovou propracovanost systému.
Kromě webového rozhraní, které slouží k úpravě a prohlížení zastávek, linek, atd. je součástí výsledného software rozhraní pro obsluhu linky v reálném čase na mobilním zařízení, 
zobrazující aktuální jízdní informace, například nejbližší zastávky a časy příjezdu k nim, aktuální zpoždění, atp.
}

%\klslova{}
%\keywords{}

\obsah
%\listoffigures
%\listoftables

\kapitola{Úvod a cíl práce}
\sekce{Úvod do problematiky}
Dnešní integrované dopravní systémy potřebují vhodný informační systém, umožňující kontrolu a řízení vozidel hromadné dopravy.
Takový \emph{telematický}\footnote{
Pojem zahrnující řadu technologií používaných hlavně v nákladní a hromadné dopravě, různá telekomunikační zařízení umožňující sdílení dat mezi vozidly a jejich sledování a řízení, satelitní navigaci apod. \cite{webfleet}
} systém je zásadním prvkem v zajišťování plynulé dopravy a maximálního komfortu pro cestující, řidiče vozidel a dispečery.
Integrované dopravní systémy obvykle používají obdobné řídicí informační systémy, poskytující například tyto výhody:
\begin{itemize}
	\item rychlá přímá komunikace mezi vozidlem a dispečinkem nebo mezi vozidly
	\item usnadnění práce řidičovi (např. itinerář jízdy, informace o změně trasy z důvodu výluk, mimořádná událost, ...)
	\item zjišťování fyzické polohy vozidel
	\item dopravní informovanost cestujících (elektronické označníky na zastávkách, aktuální poloha vozidel dostupná na internetu, ...)
	\item podkladové informace pro optimalizaci provozu (vyhodnocování jízdy vozidel podle jízdního řádu, sledování obsazenosti vozidel,...)
	\item komunikace s řadiči křižovatek pro zajištění preference vozidla hromadné dopravy
	\item pružnost při zvládání krizových a mimořádných situací
	\item příjem povelů od zrakově handicapovaných a následná akustická asistence
	\item záznam stavu vozidla a jeho okolí (např. logování diagnostických informací vozidla, palubní kamera)
\end{itemize}

Vzhledem ke stále složitějšímu provozu veřejné dopravy je propracovaný řídicí informační systém samozřejmostí současných IDS. 

\sekce{Cíl práce}
Cílem práce je navrhnout a naprogramovat funkční řídicí systém pro dopravní společnost, schopný uspokojit požadavky menšího dopravního podniku s nízkými nároky na komplexnost a celkovou propracovanost systému.
Finální program bude přístupný jako \textbf{progresivní webová aplikace} umožňujicí správu zastávek, linek, vozidel, řidičů, atd. ve webovém prohlížeči bez nutnosti instalace na zařízení uživatele. Palubní počítač vozidla bude suplován mobilním zařízením.

\begin{literatura}

\citace{webfleet}{Webfleet CZ, online}{Co je telematika? Webfleet Solutions CZ. Správa vozového parku a sledování vozidel — Webfleet Solutions CZ [online]. [cit. 2022-02-10]. Dostupné z: Co je telematika? Webfleet Solutions CZ. Správa vozového parku a sledování vozidel — Webfleet Solutions CZ [online]. [cit. 2022-02-10]. Dostupné z: \url{https://www.webfleet.com/cs_cz/webfleet/fleet-management/glossary/telematics/}}


\end{literatura}



\end{document}