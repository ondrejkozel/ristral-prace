\documentclass[twoside,12pt]{article}%
% pro tisk po jedné straně papíru je potřebné odstranit volbu twoside
\usepackage{xdippo}
% \cestina % implicitní
% \slovencina
% \english
\pismo{LModern}
%\pismo{Constantia} % nic (=LModern), Academica, Baskerville, Bookman, Cambria, Comenia, Constantia, Palatino, Times
% \dvafonty % implicitně je \jedenfont
% \technika
% \beletrie % implicitní
\popiskyzkr
% \popisky % implicitní
\pagestyle{headings} % implicitní
\cislovat{2}
\brokenpenalty 10000
\usepackage{minted}
\usepackage[hidelinks]{hyperref}
\begin{document}

\titul{Tvorba jednoduchého řídicího informačního systému dopravní společnosti}{Ondřej Kozel}{Mgr.\,Naďa Horáková}{Brno 2021}

\prohlasenimuz{V~Brně dne \today}

\abstract{Production of a simple information system for transport company. Brno, 2022.}{There will be an abstract one day!}

\abstrakt{Tvorba jednoduchého řídicího informačního systému dopravní společnosti. Závěrečná maturitní práce. Brno, 2022.}
{Práce se zabývá procesem tvorby informačního systému pro dopravní společnost schopného spravovat a řídit vozidla, zastávky, linky, řidiče a další objekty veřejné dopravy.
Jejím hlavním cílem je vytvoření produktu, který by obstál při nasazení v menším dopravním podniku s nižšími nároky na celkovou propracovanost systému.
Kromě webového rozhraní, které slouží k úpravě a prohlížení zastávek, linek, atd. je součástí výsledného software rozhraní pro obsluhu linky v reálném čase na mobilním zařízení, 
zobrazující aktuální jízdní informace, například nejbližší zastávky a časy příjezdu k nim, aktuální zpoždění, atp.
}

%\klslova{}
%\keywords{}

\obsah
\listoffigures
%\listoftables

\kapitola{Úvod a cíl práce}
\sekce{Úvod do problematiky}
Dnešní integrované dopravní systémy potřebují vhodný informační systém, umožňující kontrolu a řízení vozidel hromadné dopravy.
Takový \emph{telematický}\footnote{
Pojem zahrnující řadu technologií používaných hlavně v nákladní a hromadné dopravě, různá telekomunikační zařízení umožňující sdílení dat mezi vozidly a jejich sledování a řízení, satelitní navigaci apod. \cite{webfleet}
} systém je zásadním prvkem v zajišťování plynulé dopravy a maximálního komfortu pro cestující, řidiče vozidel a dispečery.
Integrované dopravní systémy obvykle používají obdobné řídicí informační systémy, poskytující například tyto výhody:
\begin{itemize}
	\item rychlá přímá komunikace mezi vozidlem a dispečinkem nebo mezi vozidly
	\item usnadnění práce řidičovi (např. itinerář, informace o změně trasy z důvodu výluk, mimořádná událost, ...)
	\item zjišťování fyzické polohy vozidel
	\item dopravní informovanost cestujících (elektronické označníky na zastávkách, aktuální poloha vozidel dostupná na internetu, ...)
	\item podkladové informace pro optimalizaci provozu (vyhodnocování jízdy vozidel podle jízdního řádu, sledování obsazenosti vozidel,...)
	\item komunikace s řadiči křižovatek pro zajištění preference vozidla hromadné dopravy
	\item pružnost při zvládání krizových a mimořádných situací
	\item příjem povelů od zrakově handicapovaných a následná akustická asistence
	\item záznam stavu vozidla a jeho okolí (např. logování diagnostických informací vozidla, palubní kamera)
\end{itemize}

Vzhledem ke stále složitějšímu provozu veřejné dopravy je propracovaný řídicí informační systém samozřejmostí současných IDS. 

\sekce{Cíl práce}
Cílem práce je navrhnout a naprogramovat funkční řídicí systém pro dopravní společnost, schopný uspokojit požadavky menšího dopravního podniku s nízkými nároky na komplexnost a celkovou propracovanost systému.
Finální program bude přístupný jako \textbf{progresivní webová aplikace} umožňujicí správu zastávek, linek, vozidel, řidičů, atd. ve webovém prohlížeči bez nutnosti instalace na zařízení uživatele. Palubní počítač vozidla bude suplován mobilním zařízením.

\podsekce{Plánované funkcionality systému}
Od systému je očekáváno, že bude:
\begin{itemize}
	\item pracovat ve dvou režimech:
	\begin{enumerate}
		\item režim jednotlivce – koncovým uživatelem systému je jediná osoba, která spravuje vlastní zastávky, linky, vozidla apod. a zároveň je řidičem.
		\item režim organizace – systém je využíván vícero uživateli s různými oprávněními\footnote{
Oprávnění uživatelů byla vypsána jen ve zkratce. Specifikaci naleznete v kapitole \textbf{Návrh systému}.
}:
		\begin{enumerate}
			\item \emph{řadový uživatel} – uživatel s minimálními oprávněními: pouze prohlíží zastávky, linky, vozidla apod. a svoje uskutečněné a naplánované jízdy\footnote{
\emph{Jízdou} se rozumí jediná výprava vozu na linku, postupné obsloužení zastávek od výchozí po konečnou.
}
, uskutečňuje napánované jízdy
			\item \emph{administrátor} – dědí oprávnění řadového uživatele, navíc může zastávky, linky, vozidla apod. i upravovat a plánovat jízdy řadovým uživatelům
			\item \emph{superadministrátor} – dědí oprávnění administrátora, navíc může spravovat podřízené účty (vytvářet, upravovat údaje a oprávnění, mazat)
		\end{enumerate}
	\end{enumerate}
	\item poskytovat řidiči přehledné rozhraní simulující palubní počítač na mobilním zařízení, které by mu zpřístupňovalo itinerář, informace o jízdě, případně mapu, a další související informace.
	\item umožňovat vytváření, prohlížení, úpravu a mazání zastávek, linek, vozidel a dalších objektů a jejich uložení v databázi.
	\item pomáhat s plánováním jízd a jejich přiřazováním řidičům.
	\item analyzovat proběhlé jízdy a upozorňovat na anomálie v minulých jízdách.
\end{itemize}


\kapitola{Návrh systému}

\sekce{UML}
UML, \emph{Unified Modeling Language}, je soubor grafických notací, který se používá (nejen) pří vývoji softwaru, standard v oblasti analýzy a návrhu.
Je užitečným nástrojem při plánování a vývoji software. S přibývající složitostí počítačových programů již prakticky není možné, aby celý složitý systém naprogramoval jediný člověk,
a tak bývá nutné pracovat v týmu programátorů. Mezi programátory musí probíhat velmi dobrá komunikace, kterou velmi usnadňuje používání grafických notací jako jsou například UML diagramy.
Dále může přijít vhod při odhadování ceny systému v počátcích vývoje. \cite{uvod_do_uml}

\sekce{Případy užití}
Případ užití, \emph{use case}, je sada akcí, které vedou k dosažení určitého cíle. Případem užití může být například registrace uživatele, vytvoření zastávky, zahájení jízdy apod.
– definuje funkcionality, které by měl navrhovaný systém umět. \cite{uml_use_case}

Use case je nejčastěji zakreslován jako elipsa s jeho názvem uvnitř.
V diagramu figurují ještě \emph{aktéři}, kteří představují buď uživatele, nebo nějaký externí činitel\footnote{
Externím činitelem může být například  čas. Například: v bankovní aplikaci by aktér \emph{Čas} mohl být asociovaný k UC \emph{Odeslat výpis z účtu}.
\emph{Čas} by jednou za měsíc poslal uživatelům výpis z bankovního účtu emailem.
}, kteří jsou asociovaní s vybranými use cases a tím symbolizují, že se k nim vztahují.

\obrazek
\vlozobr{edit_an_article.pdf}{1.25}
\endobr{Jednoduchý use case diagram\obrzdroj{\cite{edit_an_article}}}

\podsekce{Diagram případů užití navrhovaného systému}
\obrazekp
\vlozobrbox{ristral-use-case-diagram.png}{!}{.95\textheight}
\endobr{UML Use case diagram systému\obrzdroj{\cite{ristral_use_case}}}



\begin{literatura}

\citace{webfleet}{Webfleet CZ, Co je telematika?}{Co je telematika? Webfleet Solutions CZ. \emph{Webfleet Solutions CZ} [online]. [cit. 2022-02-10]. Dostupné z: \url{https://www.webfleet.com/cs_cz/webfleet/fleet-management/glossary/telematics/}}
\citace{uvod_do_uml}{ITnetwork.cz, Úvod do UML}{ČÁPKA, David. Úvod do UML. \emph{ITnetwork.cz: Učíme národ IT} [online]. [cit. 2022-02-10]. Dostupné z: \url{https://www.itnetwork.cz/navrh/uml/uml-uvod-historie-vyznam-a-diagramy}}
\citace{uml_use_case}{ITnetwork.cz, UML – Use Case Diagram}{ČÁPKA, David. UML - Use Case Diagram. \emph{ITnetwork.cz: Učíme národ IT} [online]. [cit. 2022-02-10]. Dostupné z: \url{https://www.itnetwork.cz/navrh/uml/uml-use-case-diagram}}

\podsekce{Obrázky}
\citace{edit_an_article}{Softzen, A very simple use case diagram of a Wiki system}{By Softzen - Own work, CC0, \url{https://commons.wikimedia.org/w/index.php?curid=32398796}}
\citace{ristral_use_case}{Ristral Use case diagram}{Vlastní tvorba. \url{doplnit odkaz}}

\end{literatura}



\end{document}